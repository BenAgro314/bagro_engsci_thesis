\documentclass{article}
\usepackage[utf8]{inputenc}
\usepackage{amsmath}
\usepackage{upgreek}
\usepackage{amsfonts}

\setcounter{MaxMatrixCols}{20}

\title{Thesis}
\author{Ben Agro}
\date{}

\begin{document}

\maketitle

\section{SDP Relaxation of Stereo Localization Problem}

\subsection{Re-write as QCQP}

\begin{align}
\mathbf{T_{cw}} = \text{argmin}_{\mathbf{T}} \sum_k (\mathbf{y}_k - \mathbf{M} \frac{1}{z_k} \mathbf{T} \mathbf{p}_k)^T \mathbf{W}_k (\mathbf{y}_k - \mathbf{M} \frac{1}{z_k} \mathbf{T} \mathbf{p}_k),\\
\mathbf{T} \in SE(3),\\
z_k = \mathbf{e}_3^T \mathbf{T} \mathbf{p}_k,
\end{align}
where $\mathbf{e}_3^T = \begin{bmatrix}0 & 0 & 1 & 0\end{bmatrix}.$
Let $\mathbf{v}_k = \frac{1}{z_k}\mathbf{T}\mathbf{p}_k$:
\begin{align}
\mathbf{v}_kz_k = \mathbf{T}\mathbf{p}_k\\
\mathbf{v}_k \mathbf{e}_3^T \mathbf{T} \mathbf{p}_k = \mathbf{T}\mathbf{p}_k\\
(\mathbf{I} - \mathbf{v}_k \mathbf{e}_3^T)\mathbf{T}\mathbf{p}_k = \mathbf{0} \in \mathbb R^4
\end{align}
Now we can re-write our optimization problem as
\begin{align}
\mathbf{T_{cw}} = \text{argmin}_{\mathbf{T}, \mathbf{v}_k} \sum_k (\mathbf{y}_k - \mathbf{M} \mathbf{v}_k)^T \mathbf{W}_k (\mathbf{y}_k - \mathbf{M} \mathbf{v}_k),\\
\mathbf{T} \in SE(3),\\
(\forall k) \quad (\mathbf{I} - \mathbf{v}_k \mathbf{e}_3^T)\mathbf{T}\mathbf{p}_k = \mathbf{0}
\end{align}
We want to write this in the standard, homogenous, form 
\begin{align}
\mathbf{x}^* = \text{argmin}_{\mathbf{x}} \quad \mathbf{x}^T \mathbf{Q} \mathbf{x}\\
(\forall i) \quad \mathbf{x}^T \mathbf{A}_i \mathbf{x} = b_i.
\end{align}

\subsubsection{Cost}

We will start with the cost, adding the homogenization variable $\omega$:
\begin{align}
\mathbf{T_{cw}} = \text{argmin}_{\mathbf{T}, \mathbf{v}_k, \omega} \sum_k (\omega \mathbf{y}_k - \mathbf{M} \mathbf{v}_k)^T \mathbf{W}_k (\omega \mathbf{y}_k - \mathbf{M} \mathbf{v}_k),\\
\mathbf{T} \in SE(3),\\
(\forall k) \quad (\mathbf{I} - \mathbf{v}_k \mathbf{a}^T)\mathbf{T}\mathbf{p}_k = \mathbf{0}\\
\omega^2 = 1.
\end{align}
We denote
\begin{align}
    \mathbf{v}_k = \frac{\mathbf{T}\mathbf{p}_k}{\mathbf{e}_3^T \mathbf{T} \mathbf{p}_k} = \begin{bmatrix}
        v_{k1} \\
        v_{k2} \\
        1 \\
        v_{k4}
    \end{bmatrix} = \begin{bmatrix}
        v_{k1} \\
        v_{k2} \\
        \omega \\
        v_{k4}
    \end{bmatrix}.
\end{align}
We add the homogenization variable so the cost and constraints remain quadtratic.

Expanding the term in the sum:
\begin{align}
(\omega\mathbf{y}_k - \mathbf{M} \mathbf{v}_k)^T \mathbf{W}_k (\omega \mathbf{y}_k - \mathbf{M} \mathbf{v}_k) = \\
\omega^2 \mathbf{y}_k^T \mathbf{W}_k \mathbf{y_k} - \omega\mathbf{y}_k^T \mathbf{W}_k \mathbf{M} \mathbf{v}_k - \omega  \mathbf{v}_k^T\mathbf{M}^T\mathbf{W}_k \mathbf{y}_k + \mathbf{v}_k^T\mathbf{M}^T\mathbf{W}_k \mathbf{M} \mathbf{v}_k \label{eq:in-sum}
\end{align}
Let
\begin{align}
    \mathbf{E} = \begin{bmatrix}
        \mathbf{e}_1 & \mathbf{e}_2 & \mathbf{e}_4
    \end{bmatrix} \in \mathbb{R}^{4 \times 3},
\end{align}
and
\begin{align}
    \mathbf{u}_k = \begin{bmatrix}
        v_{k1} \\
        v_{k2} \\
        v_{k4}
    \end{bmatrix},
\end{align}
where $\mathbf{e}_i \in \mathbb{R}^{4\times 1}$ is all zeros except for the $i^{th}$ entry which is 1.
We can then re-write $\mathbf{v}_k$:
\begin{align}
    \mathbf{v}_k = \mathbf{E}\mathbf{u}_k + \mathbf{e}_3 \omega,
\end{align}
and use this to expand equation \ref{eq:in-sum}:
\begin{align}
    &\omega^2 \mathbf{y}_k^T \mathbf{W}_k \mathbf{y}_k \\
    &- \omega \mathbf{y}_k^T \mathbf{W}_k \mathbf{M} \mathbf{E} \mathbf{u}_k - \mathbf{y}_k^T \mathbf{W}_k \mathbf{M} \mathbf{e}_3^T \omega^2 \\
    &- \omega \mathbf{u}_k^T \mathbf{E}^T \mathbf{M}^T \mathbf{W}_k \mathbf{y}_k - \mathbf{e}_3^T \mathbf{M}^T \mathbf{W}_k \mathbf{y}_k \omega^2 \\
    &+ \mathbf{u}_k \mathbf{E}^T \mathbf{M}^T \mathbf{W}_k \mathbf{M} \mathbf{E} \mathbf{u}_k + \omega \mathbf{e}_3^T \mathbf{M}^T \mathbf{W}_k \mathbf{M} \mathbf{E} \mathbf{u}_k + \mathbf{u}_k^T \mathbf{E}^T \mathbf{M}^T \mathbf{W}_k \mathbf{M} \mathbf{e}_3 \omega + \mathbf{e}_3^T \mathbf{M}^T \mathbf{W}_k \mathbf{M} \mathbf{e}_3 \omega^T.
\end{align}
Now we can write the cost in matrix form:
\begin{align}
    \mathbf{x}_2^T
    \begin{bmatrix}
        \mathbf{H}_1 & 0 & \dots & 0 & \mathbf{g}_1 \\
        0 & \mathbf{H}_2 & \dots & 0 & \mathbf{g}_2 \\
        \dots \\
        0 & 0 & \dots & \mathbf{H}_N & \mathbf{g}_N \\ 
        \mathbf{g}_1^T & \mathbf{g}_2^T & \dots &  \mathbf{g}_N^T & \mathbf{\Omega}
    \end{bmatrix}  
    \mathbf{x}_2
\end{align}
where
\begin{align}
    \mathbf{x}_2 = \begin{bmatrix} \mathbf{u}_1 \\ \mathbf{u}_2 \\ \dots \\ \mathbf{u}_N \\ \omega \end{bmatrix}, \\
    \mathbf{\Omega} = \sum_k \left( \mathbf{y}_k^T \mathbf{W}_k \mathbf{y}_k - \mathbf{y}_k^T \mathbf{W}_k \mathbf{M} \mathbf{e}_3 - \mathbf{e}_3^T \mathbf{M}^T \mathbf{W}_k \mathbf{y}_k + \mathbf{e}_3^T \mathbf{M}^T \mathbf{W}_k \mathbf{M} \mathbf{e}_3 \right) \in \mathbb{R}, \\
    \mathbf{g}_k = -\mathbf{E}^T\mathbf{M}^T\mathbf{W}_k\mathbf{y}_k + \mathbf{E}^T\mathbf{M}^T\mathbf{W}_k\mathbf{M}\mathbf{e}_3  \in \mathbb{R}^{3 \times 1}, \\
    \mathbf{H}_k = \mathbf{E}^T \mathbf{M}^T \mathbf{W}_k \mathbf{M} \mathbf{E} \in \mathbb{R}^{3 \times 3}
\end{align}

\subsubsection{Constraints}
Lets begin with the constraint $\mathbf{T} \in SE(3)$. 
Let 
\begin{align}
\mathbf{T} = \begin{bmatrix} \mathbf{C} & \mathbf{r} \\ \mathbf{0}^T & 1 \end{bmatrix}, \quad \mathbf{C} \in SO(3).\\
\mathbf{C}^T\mathbf{C} = \mathbf{I} \quad \text{and} \quad \text{det}(\mathbf{C}) = 1.
\end{align}
We will drop the determinate constraint. 
Write
\begin{align}
\mathbf{C} = \begin{bmatrix} \mathbf{c}_1 & \mathbf{c}_2 & \mathbf{c}_3 \end{bmatrix}\\
\mathbf{C}^T\mathbf{C} = \begin{bmatrix} \mathbf{c}_1^T \\ \mathbf{c}_2^T \\ \mathbf{c}_3^T \end{bmatrix} \begin{bmatrix} \mathbf{c}_1 & \mathbf{c}_2 & \mathbf{c}_3 \end{bmatrix} = \mathbf{I}
\end{align}
This implies 6 (due to symmetry we don't need all 9) quadtratic constraints in the form:
\begin{align}
\mathbf{c}_i^T \mathbf{c}_j = \delta_{i,j} = \begin{cases} 1 \quad\text{if } i = j\\ 0 \quad\text{else}\end{cases},
\end{align}
We can write these in matrix form. Let
\begin{align}
\mathbf{x}_1 = \begin{bmatrix} \mathbf{c}_1 \\ \mathbf{c}_2 \\ \mathbf{c}_3 \\ \mathbf{r} \end{bmatrix}.
\end{align}
Then the six constraints are 
\begin{align}
    \mathbf{x}_1^T  \begin{bmatrix} \mathbf{I} & \mathbf{0} & \mathbf{0} & \mathbf{0} \\ \mathbf{0} & \mathbf{0} & \mathbf{0} & \mathbf{0} \\ \mathbf{0} & \mathbf{0} & \mathbf{0} & \mathbf{0} \\ \mathbf{0} & \mathbf{0} & \mathbf{0} & \mathbf{0} \end{bmatrix} \mathbf{x}_1 = 1,\\
    \mathbf{x}_1^T  \begin{bmatrix} \mathbf{0} & \mathbf{0} & \mathbf{0} & \mathbf{0} \\ \mathbf{0} & \mathbf{I} & \mathbf{0} & \mathbf{0} \\ \mathbf{0} & \mathbf{0} & \mathbf{0} & \mathbf{0} \\ \mathbf{0} & \mathbf{0} & \mathbf{0} & \mathbf{0} \end{bmatrix} \mathbf{x}_1 = 1,\\
    \mathbf{x}_1^T  \begin{bmatrix} \mathbf{0} & \mathbf{0} & \mathbf{0} & \mathbf{0} \\ \mathbf{0} & \mathbf{0} & \mathbf{0} & \mathbf{0} \\ \mathbf{0} & \mathbf{0} & \mathbf{I} & \mathbf{0} \\ \mathbf{0} & \mathbf{0} & \mathbf{0} & \mathbf{0} \end{bmatrix} \mathbf{x}_1 = 1,\\
    \mathbf{x}_1^T  \begin{bmatrix} \mathbf{0} & \mathbf{I} & \mathbf{0} & \mathbf{0} \\ \mathbf{0} & \mathbf{0} & \mathbf{0} & \mathbf{0} \\ \mathbf{0} & \mathbf{0} & \mathbf{0} & \mathbf{0} \\ \mathbf{0} & \mathbf{0} & \mathbf{0} & \mathbf{0} \end{bmatrix} \mathbf{x}_1 = 0,\\
    \mathbf{x}_1^T  \begin{bmatrix} \mathbf{0} & \mathbf{0} & \mathbf{I} & \mathbf{0} \\ \mathbf{0} & \mathbf{0} & \mathbf{0} & \mathbf{0} \\ \mathbf{0} & \mathbf{0} & \mathbf{0} & \mathbf{0} \\ \mathbf{0} & \mathbf{0} & \mathbf{0} & \mathbf{0} \end{bmatrix} \mathbf{x}_1 = 0,\\
    \mathbf{x}_1^T  \begin{bmatrix} \mathbf{0} & \mathbf{0} & \mathbf{0} & \mathbf{0} \\ \mathbf{0} & \mathbf{0} & \mathbf{I} & \mathbf{0} \\ \mathbf{0} & \mathbf{0} & \mathbf{0} & \mathbf{0} \\ \mathbf{0} & \mathbf{0} & \mathbf{0} & \mathbf{0} \end{bmatrix} \mathbf{x}_1 = 0,\\
\end{align}

Next lets deal with the constraint $(\forall k) \quad (\mathbf{I} - \mathbf{v}_k \mathbf{e}_3^T) \mathbf{T} \mathbf{p}_k = \mathbf{0}$. Expand this and add our homogenization variable $\omega$:
\begin{align}
    (\mathbf{I} - \mathbf{v}_k \mathbf{e}_3^T) \mathbf{T} \mathbf{p}_k = \mathbf{0} \\
    \left( \begin{bmatrix}
        \omega & 0 & 0 & 0 \\
        0 & \omega & 0 & 0 \\
        0 & 0 & \omega & 0 \\ 
        0 & 0 & 0 & 1 \\
    \end{bmatrix} - \begin{bmatrix}
        v_{k1} \\
        v_{k2} \\
        \omega \\
        v_{k4} 
    \end{bmatrix} \begin{bmatrix}
        0 & 0 & 1 & 0
    \end{bmatrix} \right) \begin{bmatrix}
        \mathbf{c}_1 & \mathbf{c}_2 & \mathbf{c}_3 & \mathbf{r} \\
        0 & 0 & 0 & 1
    \end{bmatrix} \begin{bmatrix}
        p_{k1} \\
        p_{k2} \\
        p_{k3} \\
        1
    \end{bmatrix} = \mathbf{0} \\
    \begin{bmatrix}
        \omega & 0 & -v_{k1} & 0 \\
        0 & \omega & -v_{k2} & 0 \\
        0 & 0 & 0 & 0 \\
        0 & 0 & -v_{k4} & 1 \\
    \end{bmatrix} \begin{bmatrix}
        \mathbf{c}_1 p_{k1} + \mathbf{c}_2 p_{k2} + \mathbf{c}_3 p_{k3} + \mathbf{r} \\
        1
    \end{bmatrix} = \mathbf{0} \\
\end{align}
This last equation yields four three equality constraints:
{\small
\begin{align}
    \omega \mathbf{e}_1^T \mathbf{c}_1 p_{k1}
    + \omega \mathbf{e}_1^T \mathbf{c}_2 p_{k2}
    + \omega \mathbf{e}_1^T \mathbf{c}_3 p_{k3}
    + \omega \mathbf{e}_1^T \mathbf{r}
    - v_{k1}\mathbf{e}_3^T  \mathbf{c}_1 p_{k1}
    - v_{k1}\mathbf{e}_3^T \mathbf{c}_2 p_{k2}
    - v_{k1}\mathbf{e}_3^T \mathbf{c}_3 p_{k3}
    - v_{k1}\mathbf{e}_3^T \mathbf{r} = 0, \\
    \omega \mathbf{e}_2^T \mathbf{c}_1 p_{k1}
    + \omega \mathbf{e}_2^T \mathbf{c}_2 p_{k2}
    + \omega \mathbf{e}_2^T \mathbf{c}_3 p_{k3}
    + \omega \mathbf{e}_2^T \mathbf{r}
    - v_{k2}\mathbf{e}_3^T  \mathbf{c}_1 p_{k1}
    - v_{k2}\mathbf{e}_3^T \mathbf{c}_2 p_{k2}
    - v_{k2}\mathbf{e}_3^T \mathbf{c}_3 p_{k3}
    - v_{k2}\mathbf{e}_3^T \mathbf{r} = 0, \\
    v_{k4}\mathbf{e}_3^T  \mathbf{c}_1 p_{k1}
    + v_{k4}\mathbf{e}_3^T \mathbf{c}_2 p_{k2}
    + v_{k4}\mathbf{e}_3^T \mathbf{c}_3 p_{k3}
    + v_{k4}\mathbf{e}_3^T \mathbf{r} = 1, \\
\end{align}
}
where we have slightly abused notation, because in the above equation $\mathbf{e}_i \in \mathbb{R}^{3 \times 1}$.
We can write this in matrix form:
\begin{align}
    \begin{bmatrix}
        \mathbf{c}_1 \\
        \mathbf{c}_2 \\
        \mathbf{c}_3 \\
        \mathbf{r} \\
        \mathbf{u}_k \\
        \omega
    \end{bmatrix}^T
    \begin{bmatrix}
        0 & 0 & 0 & 0 & 0 & 0 \\
        0 & 0 & 0 & 0 & 0 & 0 \\
        0 & 0 & 0 & 0 & 0 & 0 \\
        0 & 0 & 0 & 0 & 0 & 0 \\
        -p_{k1}\mathbf{e}_1\mathbf{e}_3^T & -p_{k2}\mathbf{e}_1\mathbf{e}_3^T & -p_{k3}\mathbf{e}_1\mathbf{e}_3^T & -\mathbf{e}_1\mathbf{e}_3^T & 0 & 0 \\
        p_{k1}\mathbf{e}_1^T & p_{k2}\mathbf{e}_1^T & p_{k3}\mathbf{e}_1^T & \mathbf{e}_1^T & 0 &0 
    \end{bmatrix} 
    \begin{bmatrix}
        \mathbf{c}_1 \\
        \mathbf{c}_2 \\
        \mathbf{c}_3 \\
        \mathbf{r} \\
        \mathbf{u}_k \\
        \omega
    \end{bmatrix} = 0 \\
    \begin{bmatrix}
        \mathbf{c}_1 \\
        \mathbf{c}_2 \\
        \mathbf{c}_3 \\
        \mathbf{r} \\
        \mathbf{u}_k \\
        \omega
    \end{bmatrix}^T
    \begin{bmatrix}
        0 & 0 & 0 & 0 & 0 & 0 \\
        0 & 0 & 0 & 0 & 0 & 0 \\
        0 & 0 & 0 & 0 & 0 & 0 \\
        0 & 0 & 0 & 0 & 0 & 0 \\
        -p_{k1}\mathbf{e}_2\mathbf{e}_3^T & -p_{k2}\mathbf{e}_2\mathbf{e}_3^T & -p_{k3}\mathbf{e}_2\mathbf{e}_3^T & -\mathbf{e}_2\mathbf{e}_3^T & 0 & 0 \\
        p_{k1}\mathbf{e}_2^T & p_{k2}\mathbf{e}_2^T & p_{k3}\mathbf{e}_2^T & \mathbf{e}_2^T & 0 &0 
    \end{bmatrix} 
    \begin{bmatrix}
        \mathbf{c}_1 \\
        \mathbf{c}_2 \\
        \mathbf{c}_3 \\
        \mathbf{r} \\
        \mathbf{u}_k \\
        \omega
    \end{bmatrix} = 0 \\
    \begin{bmatrix}
        \mathbf{c}_1 \\
        \mathbf{c}_2 \\
        \mathbf{c}_3 \\
        \mathbf{r} \\
        \mathbf{u}_k \\
        \omega
    \end{bmatrix}^T
    \begin{bmatrix}
        0 & 0 & 0 & 0 & 0 & 0 \\
        0 & 0 & 0 & 0 & 0 & 0 \\
        0 & 0 & 0 & 0 & 0 & 0 \\
        0 & 0 & 0 & 0 & 0 & 0 \\
        p_{k1}\mathbf{e}_3\mathbf{e}_3^T & p_{k2}\mathbf{e}_3\mathbf{e}_3^T & p_{k3}\mathbf{e}_3\mathbf{e}_3^T & \mathbf{e}_3\mathbf{e}_3^T & 0 & 0 \\
        0 & 0 & 0 & 0 & 0 & 0 \\
    \end{bmatrix} 
    \begin{bmatrix}
        \mathbf{c}_1 \\
        \mathbf{c}_2 \\
        \mathbf{c}_3 \\
        \mathbf{r} \\
        \mathbf{u}_k \\
        \omega
    \end{bmatrix} = 1 
\end{align}

\subsection{SDP Relaxation}

Now that we have the QCQP in the homogenous form
\begin{align}
\mathbf{x}^* = \text{argmin}_{\mathbf{x}} \quad \mathbf{x}^T \mathbf{Q} \mathbf{x}\\
(\forall i) \quad \mathbf{x}^T \mathbf{A}_i \mathbf{x} = b_i,
\end{align}
we can easily turn it into an SDP:
\begin{align}
\mathbf{x}^* = \text{argmin}_{\mathbf{x}} \quad \text{tr}(\mathbf{Q}\mathbf{x}\mathbf{x}^T),\\
\text{s.t}\quad (\forall i) \quad \text{tr}(\mathbf{A}_i \mathbf{x} \mathbf{x}^T) = b_i.
\end{align}
Let $\mathbf{X} = \mathbf{x}\mathbf{x}^T \in \mathbb{R}^{(14 + 4n) \times (14 + 4n)}$, then the above is equivalent to:
\begin{align}
\mathbf{X}^* = \text{argmin}_{\mathbf{X}} \quad \text{tr}(\mathbf{Q}\mathbf{X}),\\
\text{s.t}\quad (\forall i) \quad \text{tr}(\mathbf{A}_i \mathbf{X}) = b_i,\\
\mathbf{X} = \mathbf{x} \mathbf{x}^T\\
\text{rank}(\mathbf{X}) = 1
\end{align}
We can relax these last two constraints to get an SDP:
\begin{align}
\mathbf{X}^* = \text{argmin}_{\mathbf{X}} \quad \text{tr}(\mathbf{Q}\mathbf{X}),\\
\text{s.t}\quad (\forall i) \quad \text{tr}(\mathbf{A}_i \mathbf{X}) = b_i,\\
\mathbf{X} \succeq 0.
\end{align}

\end{document}