\documentclass{article}
\usepackage[utf8]{inputenc}
\usepackage{amsmath}
\usepackage{bm}
\usepackage{amsfonts}

\setcounter{MaxMatrixCols}{20}

\title{Thesis}
\author{Ben Agro}
\date{}

\begin{document}

\maketitle

\section{SDP Relaxation of Stereo Localization Problem}

\subsection{Re-write as QCQP}

\begin{align}
\mathbf{T_{cw}} = \text{argmin}_{\mathbf{T}} \sum_k (\mathbf{y}_k - \mathbf{M} \frac{1}{z_k} \mathbf{T} \mathbf{p}_k)^T \mathbf{W}_k (\mathbf{y}_k - \mathbf{M} \frac{1}{z_k} \mathbf{T} \mathbf{p}_k),\\
\mathbf{T} \in SE(3),\\
z_k = \mathbf{a}^T \mathbf{T} \mathbf{p}_k,
\end{align}
where $\mathbf{a}^T = \begin{bmatrix}0 & 0 & 1 & 0\end{bmatrix}.$
Let $\mathbf{v}_k = \frac{1}{z_k}\mathbf{T}\mathbf{p}_k$:
\begin{align}
\mathbf{v}_kz_k = \mathbf{T}\mathbf{p}_k\\
\mathbf{v}_k \mathbf{a}^T \mathbf{T} \mathbf{p}_k = \mathbf{T}\mathbf{p}_k\\
(\mathbf{I} - \mathbf{v}_k \mathbf{a}^T)\mathbf{T}\mathbf{p}_k = \mathbf{0} \in \mathbb R^4
\end{align}
Now we can re-write our optimization problem as
\begin{align}
\mathbf{T_{cw}} = \text{argmin}_{\mathbf{T}, \mathbf{v}_k} \sum_k (\mathbf{y}_k - \mathbf{M} \mathbf{v}_k)^T \mathbf{W}_k (\mathbf{y}_k - \mathbf{M} \mathbf{v}_k),\\
\mathbf{T} \in SE(3),\\
(\forall k) \quad (\mathbf{I} - \mathbf{v}_k \mathbf{a}^T)\mathbf{T}\mathbf{p}_k = \mathbf{0}
\end{align}
We want to write this in the standard, homogenous, form 
\begin{align}
\mathbf{x}^* = \text{argmin}_{\mathbf{x}} \quad \mathbf{x}^T \mathbf{Q} \mathbf{x}\\
(\forall i) \quad \mathbf{x}^T \mathbf{A}_i \mathbf{x} = b_i.
\end{align}

\subsubsection{Cost}

We will start with the cost, adding the homogenization variable $\omega_0$:
\begin{align}
\mathbf{T_{cw}} = \text{argmin}_{\mathbf{T}, \mathbf{v}_k, \omega_0} \sum_k (\omega_0\mathbf{y}_k - \mathbf{M} \mathbf{v}_k)^T \mathbf{W}_k (\omega_0 \mathbf{y}_k - \mathbf{M} \mathbf{v}_k),\\
\mathbf{T} \in SE(3),\\
(\forall k) \quad (\mathbf{I} - \mathbf{v}_k \mathbf{a}^T)\mathbf{T}\mathbf{p}_k = \mathbf{0}\\
\omega_0^2 = 1.
\end{align}
Expanding the term in the sum:
\begin{align}
(\omega_0\mathbf{y}_k - \mathbf{M} \mathbf{v}_k)^T \mathbf{W}_k (\omega_0 \mathbf{y}_k - \mathbf{M} \mathbf{v}_k) = \\
\omega_0^2 \mathbf{y}_k^T \mathbf{W}_k \mathbf{y_k} - \omega_0\mathbf{y}_k^T \mathbf{W}_k \mathbf{M} \mathbf{v}_k - \omega_0  \mathbf{v}_k^T\mathbf{M}^T\mathbf{W}_k \mathbf{y}_k + \mathbf{v}_k^T\mathbf{M}^T\mathbf{W}_k \mathbf{M} \mathbf{v}_k\\
=
\begin{bmatrix} 
    \mathbf{v}_k \\
    \omega_0
\end{bmatrix}^T
\begin{bmatrix}
    \mathbf{M}^T\mathbf{W}_k \mathbf{M}     & -\mathbf{M}^T \mathbf{W}_k \mathbf{y}_k \\
    -\mathbf{y}_k^T \mathbf{W}_k \mathbf{M} & \mathbf{y}^T \mathbf{W}_k \mathbf{y}
\end{bmatrix}
\begin{bmatrix}
    \mathbf{v}_k \\
    \omega_0
\end{bmatrix}.
\end{align}
We can extend this to express the whole some in quadtratic form:
\begin{align}
\sum_k (\omega_0\mathbf{y}_k - \mathbf{M} \mathbf{v}_k)^T \mathbf{W}_k (\omega_0 \mathbf{y}_k - \mathbf{M} \mathbf{v}_k) = \\
\begin{bmatrix}
    \mathbf{v}_1 \\
    \dots \\
    \mathbf{v}_n \\
    \omega_0
\end{bmatrix}^T
\begin{bmatrix}
    \mathbf{M}^T\mathbf{W}_1 \mathbf{M} & 0 & \dots & 0 & -\mathbf{M}^T\mathbf{W}_1\mathbf{y}_1 \\
    0 & \mathbf{M}^T \mathbf{W}_2 \mathbf{M} & \dots & 0 & -\mathbf{M}^T\mathbf{W}_2\mathbf{y}_2 \\
    \dots \\
    0 & 0 & \dots & \mathbf{M}^T \mathbf{W}_n \mathbf{M} & -\mathbf{M}^T \mathbf{W}_n \mathbf{y}_n \\
    -\mathbf{y}_1^T \mathbf{W}_1 \mathbf{M} & -\mathbf{y}_2^T \mathbf{W}_2^T \mathbf{M} & \dots & -\mathbf{y}_n^T \mathbf{W}_n^T \mathbf{M} & \sum_k \mathbf{y}_k^T \mathbf{W}_k \mathbf{y}_k
\end{bmatrix}
\begin{bmatrix}
    \mathbf{v}_1 \\
    \dots \\
    \mathbf{v}_n \\
    \omega_0
\end{bmatrix}
\end{align}

\subsubsection{Constraints}
Lets begin with the constraint $\mathbf{T} \in SE(3)$. 
Let 
\begin{align}
\mathbf{T} = \begin{bmatrix} \mathbf{C} & \mathbf{r} \\ \mathbf{0}^T & 1 \end{bmatrix}, \quad \mathbf{C} \in SO(3).\\
\mathbf{C}^T\mathbf{C} = \mathbf{I} \quad \text{and} \quad \text{det}(\mathbf{C}) = 1.
\end{align}
We will drop the determinate constraint. 
Write
\begin{align}
\mathbf{C} = \begin{bmatrix} \mathbf{c}_1 & \mathbf{c}_2 & \mathbf{c}_3 \end{bmatrix}\\
\mathbf{C}^T\mathbf{C} = \begin{bmatrix} \mathbf{c}_1^T \\ \mathbf{c}_2^T \\ \mathbf{c}_3^T \end{bmatrix} \begin{bmatrix} \mathbf{c}_1 & \mathbf{c}_2 & \mathbf{c}_3 \end{bmatrix} = \mathbf{I}
\end{align}
This implies 6 (due to symmetry we don't need all 9) quadtratic constraints in the form:
\begin{align}
\mathbf{c}_i^T \mathbf{c}_j = \delta_{i,j} = \begin{cases} 1 \quad\text{if } i = j\\ 0 \quad\text{else}\end{cases},
\end{align}
We can write these in matrix form. Let
\begin{align}
\mathbf{x}_1 = \begin{bmatrix} \mathbf{c}_1 \\ \mathbf{c}_2 \\ \mathbf{c}_3 \\ \mathbf{r} \end{bmatrix}.
\end{align}
Then the six constraints are 
\begin{align}
    \mathbf{x}_1^T  \begin{bmatrix} 1 & 0 & 0 & 0 \\ 0 & 0 & 0 & 0 \\ 0 & 0 & 0 & 0 \\ 0 & 0 & 0 & 0 \end{bmatrix} \mathbf{x}_1 = 1,\\
    \mathbf{x}_1^T  \begin{bmatrix} 0 & 0 & 0 & 0 \\ 0 & 1 & 0 & 0 \\ 0 & 0 & 0 & 0 \\ 0 & 0 & 0 & 0 \end{bmatrix} \mathbf{x}_1 = 1,\\
    \mathbf{x}_1^T  \begin{bmatrix} 0 & 0 & 0 & 0 \\ 0 & 0 & 0 & 0 \\ 0 & 0 & 1 & 0 \\ 0 & 0 & 0 & 0 \end{bmatrix} \mathbf{x}_1 = 1,\\
    \mathbf{x}_1^T  \begin{bmatrix} 0 & 1 & 0 & 0 \\ 0 & 0 & 0 & 0 \\ 0 & 0 & 0 & 0 \\ 0 & 0 & 0 & 0 \end{bmatrix} \mathbf{x}_1 = 0,\\
    \mathbf{x}_1^T  \begin{bmatrix} 0 & 0 & 1 & 0 \\ 0 & 0 & 0 & 0 \\ 0 & 0 & 0 & 0 \\ 0 & 0 & 0 & 0 \end{bmatrix} \mathbf{x}_1 = 0,\\
    \mathbf{x}_1^T  \begin{bmatrix} 0 & 0 & 0 & 0 \\ 0 & 0 & 1 & 0 \\ 0 & 0 & 0 & 0 \\ 0 & 0 & 0 & 0 \end{bmatrix} \mathbf{x}_1 = 0,\\
\end{align}
Next lets deal with the constraint $(\forall k) \quad (\mathbf{I} - \mathbf{v}_k \mathbf{a}^T) \mathbf{T} \mathbf{p}_k = \mathbf{0}$. Expand this and add our homogenization variable $\omega_0$:
\begin{align}
    \begin{bmatrix} \omega_0 & 0 & 0 & 0 \\ 0 & \omega_0 & 0 & 0 \\ 0 & 0 & \omega_0 & 0 \\ 0 & 0 & 0 & 1\end{bmatrix}\begin{bmatrix} \mathbf{c}_1 & \mathbf{c}_2 & \mathbf{c}_3 & \mathbf{r} \\ 0 & 0 & 0 & 1\end{bmatrix} \begin{bmatrix} p_{k1} \\ p_{k2} \\ p_{k3} \\ 1 \end{bmatrix} - \mathbf{v}_k \begin{bmatrix} 0 & 0 & 1 & 0 \end{bmatrix} \begin{bmatrix} \mathbf{c}_1 & \mathbf{c}_2 & \mathbf{c}_3 & \mathbf{r} \\ 0 & 0 & 0 & 1\end{bmatrix} \begin{bmatrix} p_{k1} \\ p_{k2} \\ p_{k3} \\ 1 \end{bmatrix} = \mathbf{0},
\end{align}
with the added constraint:
\begin{align}
    \omega_0^2 = 1.
\end{align}
Expanding further:
\begin{align}
    \begin{bmatrix}\omega_0 p_{k1} \mathbf{c}_1 + \omega_0 p_{k2} \mathbf{c}_2 + \omega_0 p_{k3} \mathbf{c}_3 + \omega_0 \mathbf{r} \\ 1\end{bmatrix} - \mathbf{v}_k \begin{bmatrix} 0 & 0 & 1 & 0 \end{bmatrix} \begin{bmatrix} p_{k1} \mathbf{c}_1 + p_{k2} \mathbf{c}_2 + p_{k3} \mathbf{c}_3 + \mathbf{r} \\ 1\end{bmatrix} = \mathbf{0}.
\end{align}
Note that
\begin{align}
    \mathbf{C} = \begin{bmatrix} \mathbf{c}_1 & \mathbf{c}_2 & \mathbf{c}_3\end{bmatrix} = \begin{bmatrix} c_{11} & c_{12} & c_{13} \\ c_{21} & c_{22} & c_{23} \\ c_{31} & c_{32} & c_{33} \end{bmatrix},
\end{align}
so we can expand the expression above:
\begin{align}
    \begin{bmatrix}\omega_0 p_{k1} \mathbf{c}_1 + \omega_0 p_{k2} \mathbf{c}_2 + \omega_0 p_{k3} \mathbf{c}_3 + \omega_0 \mathbf{r} \\ 1\end{bmatrix} - \mathbf{v}_k (p_{k1}c_{31} + p_{k2} c_{32} + p_{k3} c_{33} + r_3) = \mathbf{0}.
\end{align}
Write this as four scalar equations:
\begin{align}
    p_{k1} \omega_0 c_{11} + p_{k2} \omega_0 c_{12} + p_{k3} \omega_0 c_{13} + \omega_0 r_1 - p_{k1} c_{31} v_{k1}- p_{k2} c_{32} v_{k1} - p_{k3} c_{33} v_{k1} - r_3 v_{k1}= 0,\\
    p_{k1} \omega_0 c_{21} + p_{k2} \omega_0 c_{22} + p_{k3} \omega_0 c_{23} + \omega_0 r_1 - p_{k1} c_{31} v_{k2}- p_{k2} c_{32} v_{k2} - p_{k3} c_{33} v_{k2} - r_3 v_{k2}= 0,\\
    p_{k1} \omega_0 c_{31} + p_{k2} \omega_0 c_{32} + p_{k3} \omega_0 c_{33} + \omega_0 r_1 - p_{k1} c_{31} v_{k3}- p_{k2} c_{32} v_{k3} - p_{k3} c_{33} v_{k3} - r_3 v_{k3}= 0,\\
    1 - p_{k1} c_{31} v_{k4}- p_{k2} c_{32} v_{k4} - p_{k3} c_{33} v_{k4} - r_3 v_{k4}= 0.
\end{align}
Now we can turn these into four matrix form:
\begin{align}
    \begin{bmatrix}
        \mathbf{c}_1 \\
        \mathbf{c}_2 \\
        \mathbf{c}_3 \\
        \mathbf{r} \\
        \mathbf{v}_k \\
        \omega_0
    \end{bmatrix}^T 
    \begin{bmatrix}
        0 & 0 & 0 & 0 & 0 & 0 & 0 & 0 & 0 & 0 & 0 & 0 & 0 & 0 & 0 & 0 & 0 \\
        0 & 0 & 0 & 0 & 0 & 0 & 0 & 0 & 0 & 0 & 0 & 0 & 0 & 0 & 0 & 0 & 0 \\
        0 & 0 & 0 & 0 & 0 & 0 & 0 & 0 & 0 & 0 & 0 & 0 & -p_{k1} & 0 & 0 & 0 & 0 \\
        0 & 0 & 0 & 0 & 0 & 0 & 0 & 0 & 0 & 0 & 0 & 0 & 0 & 0 & 0 & 0 & 0 \\
        0 & 0 & 0 & 0 & 0 & 0 & 0 & 0 & 0 & 0 & 0 & 0 & 0 & 0 & 0 & 0 & 0 \\
        0 & 0 & 0 & 0 & 0 & 0 & 0 & 0 & 0 & 0 & 0 & 0 & -p_{k2} & 0 & 0 & 0 & 0 \\
        0 & 0 & 0 & 0 & 0 & 0 & 0 & 0 & 0 & 0 & 0 & 0 & 0 & 0 & 0 & 0 & 0 \\
        0 & 0 & 0 & 0 & 0 & 0 & 0 & 0 & 0 & 0 & 0 & 0 & 0 & 0 & 0 & 0 & 0 \\
        0 & 0 & 0 & 0 & 0 & 0 & 0 & 0 & 0 & 0 & 0 & 0 & -p_{k3} & 0 & 0 & 0 & 0 \\
        0 & 0 & 0 & 0 & 0 & 0 & 0 & 0 & 0 & 0 & 0 & 0 & 0 & 0 & 0 & 0 & 0 \\
        0 & 0 & 0 & 0 & 0 & 0 & 0 & 0 & 0 & 0 & 0 & 0 & 0 & 0 & 0 & 0 & 0 \\
        0 & 0 & 0 & 0 & 0 & 0 & 0 & 0 & 0 & 0 & 0 & 0 & -1 & 0 & 0 & 0 & 0 \\
        0 & 0 & 0 & 0 & 0 & 0 & 0 & 0 & 0 & 0 & 0 & 0 & 0 & 0 & 0 & 0 & 0 \\
        0 & 0 & 0 & 0 & 0 & 0 & 0 & 0 & 0 & 0 & 0 & 0 & 0 & 0 & 0 & 0 & 0 \\
        0 & 0 & 0 & 0 & 0 & 0 & 0 & 0 & 0 & 0 & 0 & 0 & 0 & 0 & 0 & 0 & 0 \\
        0 & 0 & 0 & 0 & 0 & 0 & 0 & 0 & 0 & 0 & 0 & 0 & 0 & 0 & 0 & 0 & 0 \\
        p_{k1} & 0 & 0 & p_{k2} & 0 & 0 & p_{k3} & 0 & 0 & 1 & 0 & 0 & 0 & 0 & 0 & 0 & 0
    \end{bmatrix}
    \begin{bmatrix}
        \mathbf{c}_1 \\
        \mathbf{c}_2 \\
        \mathbf{c}_3 \\
        \mathbf{r} \\
        \mathbf{v}_k \\
        \omega_0
    \end{bmatrix} = 0, \\
    \begin{bmatrix}
        \mathbf{c}_1 \\
        \mathbf{c}_2 \\
        \mathbf{c}_3 \\
        \mathbf{r} \\
        \mathbf{v}_k \\
        \omega_0
    \end{bmatrix}^T 
    \begin{bmatrix}
        0 & 0 & 0 & 0 & 0 & 0 & 0 & 0 & 0 & 0 & 0 & 0 & 0 & 0 & 0 & 0 & 0 \\
        0 & 0 & 0 & 0 & 0 & 0 & 0 & 0 & 0 & 0 & 0 & 0 & 0 & 0 & 0 & 0 & 0 \\
        0 & 0 & 0 & 0 & 0 & 0 & 0 & 0 & 0 & 0 & 0 & 0 & 0 & -p_{k1} & 0 & 0 & 0 \\
        0 & 0 & 0 & 0 & 0 & 0 & 0 & 0 & 0 & 0 & 0 & 0 & 0 & 0 & 0 & 0 & 0 \\
        0 & 0 & 0 & 0 & 0 & 0 & 0 & 0 & 0 & 0 & 0 & 0 & 0 & 0 & 0 & 0 & 0 \\
        0 & 0 & 0 & 0 & 0 & 0 & 0 & 0 & 0 & 0 & 0 & 0 & 0 & -p_{k2} & 0 & 0 & 0 \\
        0 & 0 & 0 & 0 & 0 & 0 & 0 & 0 & 0 & 0 & 0 & 0 & 0 & 0 & 0 & 0 & 0 \\
        0 & 0 & 0 & 0 & 0 & 0 & 0 & 0 & 0 & 0 & 0 & 0 & 0 & 0 & 0 & 0 & 0 \\
        0 & 0 & 0 & 0 & 0 & 0 & 0 & 0 & 0 & 0 & 0 & 0 & 0 & -p_{k3} & 0 & 0 & 0 \\
        0 & 0 & 0 & 0 & 0 & 0 & 0 & 0 & 0 & 0 & 0 & 0 & 0 & 0 & 0 & 0 & 0 \\
        0 & 0 & 0 & 0 & 0 & 0 & 0 & 0 & 0 & 0 & 0 & 0 & 0 & 0 & 0 & 0 & 0 \\
        0 & 0 & 0 & 0 & 0 & 0 & 0 & 0 & 0 & 0 & 0 & 0 & 0 & -1 & 0 & 0 & 0 \\
        0 & 0 & 0 & 0 & 0 & 0 & 0 & 0 & 0 & 0 & 0 & 0 & 0 & 0 & 0 & 0 & 0 \\
        0 & 0 & 0 & 0 & 0 & 0 & 0 & 0 & 0 & 0 & 0 & 0 & 0 & 0 & 0 & 0 & 0 \\
        0 & 0 & 0 & 0 & 0 & 0 & 0 & 0 & 0 & 0 & 0 & 0 & 0 & 0 & 0 & 0 & 0 \\
        0 & 0 & 0 & 0 & 0 & 0 & 0 & 0 & 0 & 0 & 0 & 0 & 0 & 0 & 0 & 0 & 0 \\
        0 & p_{k1} & 0 & 0 & p_{k2} & 0 & 0 & p_{k3} & 0 & 0 & 1 & 0 & 0 & 0 & 0 & 0 & 0
    \end{bmatrix}
    \begin{bmatrix}
        \mathbf{c}_1 \\
        \mathbf{c}_2 \\
        \mathbf{c}_3 \\
        \mathbf{r} \\
        \mathbf{v}_k \\
        \omega_0
    \end{bmatrix} = 0,
\end{align}
\begin{align}
    \begin{bmatrix}
        \mathbf{c}_1 \\
        \mathbf{c}_2 \\
        \mathbf{c}_3 \\
        \mathbf{r} \\
        \mathbf{v}_k \\
        \omega_0
    \end{bmatrix}^T 
    \begin{bmatrix}
        0 & 0 & 0 & 0 & 0 & 0 & 0 & 0 & 0 & 0 & 0 & 0 & 0 & 0 & 0 & 0 & 0 \\
        0 & 0 & 0 & 0 & 0 & 0 & 0 & 0 & 0 & 0 & 0 & 0 & 0 & 0 & 0 & 0 & 0 \\
        0 & 0 & 0 & 0 & 0 & 0 & 0 & 0 & 0 & 0 & 0 & 0 & 0 & 0 & -p_{k1} & 0 & 0 \\
        0 & 0 & 0 & 0 & 0 & 0 & 0 & 0 & 0 & 0 & 0 & 0 & 0 & 0 & 0 & 0 & 0 \\
        0 & 0 & 0 & 0 & 0 & 0 & 0 & 0 & 0 & 0 & 0 & 0 & 0 & 0 & 0 & 0 & 0 \\
        0 & 0 & 0 & 0 & 0 & 0 & 0 & 0 & 0 & 0 & 0 & 0 & 0 & 0 & -p_{k2} & 0 & 0 \\
        0 & 0 & 0 & 0 & 0 & 0 & 0 & 0 & 0 & 0 & 0 & 0 & 0 & 0 & 0 & 0 & 0 \\
        0 & 0 & 0 & 0 & 0 & 0 & 0 & 0 & 0 & 0 & 0 & 0 & 0 & 0 & 0 & 0 & 0 \\
        0 & 0 & 0 & 0 & 0 & 0 & 0 & 0 & 0 & 0 & 0 & 0 & 0 & 0 & -p_{k3} & 0 & 0 \\
        0 & 0 & 0 & 0 & 0 & 0 & 0 & 0 & 0 & 0 & 0 & 0 & 0 & 0 & 0 & 0 & 0 \\
        0 & 0 & 0 & 0 & 0 & 0 & 0 & 0 & 0 & 0 & 0 & 0 & 0 & 0 & 0 & 0 & 0 \\
        0 & 0 & 0 & 0 & 0 & 0 & 0 & 0 & 0 & 0 & 0 & 0 & 0 & 0 & -1 & 0 & 0 \\
        0 & 0 & 0 & 0 & 0 & 0 & 0 & 0 & 0 & 0 & 0 & 0 & 0 & 0 & 0 & 0 & 0 \\
        0 & 0 & 0 & 0 & 0 & 0 & 0 & 0 & 0 & 0 & 0 & 0 & 0 & 0 & 0 & 0 & 0 \\
        0 & 0 & 0 & 0 & 0 & 0 & 0 & 0 & 0 & 0 & 0 & 0 & 0 & 0 & 0 & 0 & 0 \\
        0 & 0 & 0 & 0 & 0 & 0 & 0 & 0 & 0 & 0 & 0 & 0 & 0 & 0 & 0 & 0 & 0 \\
        0 & 0 & p_{k1} & 0 & 0 & p_{k2} & 0 & 0 & p_{k3} & 0 & 0 & 1 & 0 & 0 & 0 & 0 & 0
    \end{bmatrix}
    \begin{bmatrix}
        \mathbf{c}_1 \\
        \mathbf{c}_2 \\
        \mathbf{c}_3 \\
        \mathbf{r} \\
        \mathbf{v}_k \\
        \omega_0
    \end{bmatrix} = 0, \\
    \begin{bmatrix}
        \mathbf{c}_1 \\
        \mathbf{c}_2 \\
        \mathbf{c}_3 \\
        \mathbf{r} \\
        \mathbf{v}_k \\
        \omega_0
    \end{bmatrix}^T 
    \begin{bmatrix}
        0 & 0 & 0 & 0 & 0 & 0 & 0 & 0 & 0 & 0 & 0 & 0 & 0 & 0 & 0 & 0 & 0 \\
        0 & 0 & 0 & 0 & 0 & 0 & 0 & 0 & 0 & 0 & 0 & 0 & 0 & 0 & 0 & 0 & 0 \\
        0 & 0 & 0 & 0 & 0 & 0 & 0 & 0 & 0 & 0 & 0 & 0 & 0 & 0 & 0 & p_{k1} & 0 \\
        0 & 0 & 0 & 0 & 0 & 0 & 0 & 0 & 0 & 0 & 0 & 0 & 0 & 0 & 0 & 0 & 0 \\
        0 & 0 & 0 & 0 & 0 & 0 & 0 & 0 & 0 & 0 & 0 & 0 & 0 & 0 & 0 & 0 & 0 \\
        0 & 0 & 0 & 0 & 0 & 0 & 0 & 0 & 0 & 0 & 0 & 0 & 0 & 0 & 0 & p_{k2} & 0 \\
        0 & 0 & 0 & 0 & 0 & 0 & 0 & 0 & 0 & 0 & 0 & 0 & 0 & 0 & 0 & 0 & 0 \\
        0 & 0 & 0 & 0 & 0 & 0 & 0 & 0 & 0 & 0 & 0 & 0 & 0 & 0 & 0 & 0 & 0 \\
        0 & 0 & 0 & 0 & 0 & 0 & 0 & 0 & 0 & 0 & 0 & 0 & 0 & 0 & 0 & p_{k3} & 0 \\
        0 & 0 & 0 & 0 & 0 & 0 & 0 & 0 & 0 & 0 & 0 & 0 & 0 & 0 & 0 & 0 & 0 \\
        0 & 0 & 0 & 0 & 0 & 0 & 0 & 0 & 0 & 0 & 0 & 0 & 0 & 0 & 0 & 0 & 0 \\
        0 & 0 & 0 & 0 & 0 & 0 & 0 & 0 & 0 & 0 & 0 & 0 & 0 & 0 & 0 & 1 & 0 \\
        0 & 0 & 0 & 0 & 0 & 0 & 0 & 0 & 0 & 0 & 0 & 0 & 0 & 0 & 0 & 0 & 0 \\
        0 & 0 & 0 & 0 & 0 & 0 & 0 & 0 & 0 & 0 & 0 & 0 & 0 & 0 & 0 & 0 & 0 \\
        0 & 0 & 0 & 0 & 0 & 0 & 0 & 0 & 0 & 0 & 0 & 0 & 0 & 0 & 0 & 0 & 0 \\
        0 & 0 & 0 & 0 & 0 & 0 & 0 & 0 & 0 & 0 & 0 & 0 & 0 & 0 & 0 & 0 & 0 \\
        0 & 0 & 0 & 0 & 0 & 0 & 0 & 0 & 0 & 0 & 0 & 0 & 0 & 0 & 0 & 0 & 0
    \end{bmatrix}
    \begin{bmatrix}
        \mathbf{c}_1 \\
        \mathbf{c}_2 \\
        \mathbf{c}_3 \\
        \mathbf{r} \\
        \mathbf{v}_k \\
        \omega_0
    \end{bmatrix} = 1,
\end{align}
Finally, the constraints for the homogenous variable is 
\begin{align}
    \omega_0 \begin{bmatrix} 1 \end{bmatrix} \omega_0 = 1,\\
\end{align}
Let
\begin{align}
\mathbf{x} = \begin{bmatrix} \mathbf{c}_1 \\ \mathbf{c}_2 \\ \mathbf{c}_3 \\ \mathbf{r} \\ \mathbf{v}_1 \\ \mathbf{v}_2 \\ \dots \\ \mathbf{v}_n \\ \omega_0 \\ \omega_0 \end{bmatrix} \in \mathbb{R}^{(14 + 4n) \times 1},
\end{align}
which includes all the decision variables. 
All of the $9 + 4 + 2 = 15$ constraints above can be formulated using $\mathbf{x}$ by adding columns and rows of zero to the matrices above to obtain $\mathbf{A}_i$,
to ignore the decision variables that are not used in the constraint. Any non-symmetric $\mathbf{A}_i$ can be made symmetric and preserve the same constraint by doing
\begin{align}
    \mathbf{A}_{\text{new}_i} = \frac{1}{2}(\mathbf{A}_i + \mathbf{A}_i^T).
\end{align}

\subsection{SDP Relaxation}

Now that we have the QCQP in the homogenous form
\begin{align}
\mathbf{x}^* = \text{argmin}_{\mathbf{x}} \quad \mathbf{x}^T \mathbf{Q} \mathbf{x}\\
(\forall i) \quad \mathbf{x}^T \mathbf{A}_i \mathbf{x} = b_i,
\end{align}
we can easily turn it into an SDP:
\begin{align}
\mathbf{x}^* = \text{argmin}_{\mathbf{x}} \quad \text{tr}(\mathbf{Q}\mathbf{x}\mathbf{x}^T),\\
\text{s.t}\quad (\forall i) \quad \text{tr}(\mathbf{A}_i \mathbf{x} \mathbf{x}^T) = b_i.
\end{align}
Let $\mathbf{X} = \mathbf{x}\mathbf{x}^T \in \mathbb{R}^{(14 + 4n) \times (14 + 4n)}$, then the above is equivalent to:
\begin{align}
\mathbf{X}^* = \text{argmin}_{\mathbf{X}} \quad \text{tr}(\mathbf{Q}\mathbf{X}),\\
\text{s.t}\quad (\forall i) \quad \text{tr}(\mathbf{A}_i \mathbf{X}) = b_i,\\
\mathbf{X} = \mathbf{x} \mathbf{x}^T\\
\text{rank}(\mathbf{X}) = 1
\end{align}
We can relax these last two constraints to get an SDP:
\begin{align}
\mathbf{X}^* = \text{argmin}_{\mathbf{X}} \quad \text{tr}(\mathbf{Q}\mathbf{X}),\\
\text{s.t}\quad (\forall i) \quad \text{tr}(\mathbf{A}_i \mathbf{X}) = b_i,\\
\mathbf{X} \succeq 0.
\end{align}

\end{document}